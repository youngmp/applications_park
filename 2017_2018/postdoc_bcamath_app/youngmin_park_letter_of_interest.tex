\documentclass[a4paper,11pt]{article}
\usepackage[utf8]{inputenc}

% extra packages
\usepackage{amsmath, amssymb}
\usepackage{graphicx}
\usepackage{color}
\usepackage{hyperref}

%\setlength{\parindent}{0pt}

\begin{document}

\begin{center}
\Large \textbf{Letter of Interest}

\Large Youngmin Park
\end{center}

I have a strong interest in working with Dr. Rodrigues, as his research is intriguing, impactful, and relevant to my interests. I particularly enjoy the direct usage of mean field models and bifurcation analyses to describe experimental observations \cite{rodrigues2009transitions,breakspear2005unifying}. While I have experience in analyzing neural models through analytical and numerical techniques \cite{park2016weakly,park2016infinitesimal} including a mean-field cortical model first studied in \cite{pinto_ermentrout_2001_siam}, my research has not yet afforded the opportunity to use experimental data to motivate the analysis.

Other publications by Dr. Rodrigues coincide strongly with my research. In \cite{rodrigues2006genesis}, the authors discuss the transitions from oscillatory solutions to spike-wave solutions in a thalamic circuitry component of a mean-field model. Although I am unfamiliar with the biology of this particular problem, I recognize that the model reduction and piecewise linear approximation make the analysis tractable.

In \cite{rodrigues2010mappings}, the authors consider mappings from a macroscopic cortical model to microscoptic conductance-based models. I find the generality of the models considered intriguing, as well as the authors' care to explicitly state the limitations of each assumption. As mentioned in my research statement, I have some interest in this type of analysis.

To summarize, I appreciate the biologically motivated problems that Dr. Rodrigues and colleagues address without using a purely statistical approach. If I were to join this group, I would be able to satisfy three important personal goals: first, to continue learning more dynamical systems, second, to apply the knowledge I learned throughout my doctorate, and third, to learn more neuroscience.

\bibliographystyle{plain}
\bibliography{../../youngmin-bard/bibliography,../../youngmin-bard/ymp,../../youngmin-bard/bio,../../youngmin-bard/neuralfield,../../youngmin-bard/mean_field}


\end{document}
