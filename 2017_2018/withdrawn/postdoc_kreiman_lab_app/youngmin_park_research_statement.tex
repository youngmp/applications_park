\documentclass[a4paper,11pt]{article}
\usepackage[utf8]{inputenc}

% extra packages
\usepackage{amsmath, amssymb}
\usepackage{graphicx}
\usepackage{color}
\usepackage{hyperref}

%\setlength{\parindent}{0pt}

\begin{document}

\begin{center}
\Large \textbf{Statement of Research Interests}

\Large Youngmin Park
\end{center}


\section{Background}
Generally, I am interested in learning the mathematics necessary to address problems that arise in neuroscience. In practice, my research is largely focused on the study of deterministic, smooth and non-smooth dynamical systems, especially systems that arise in mathematical neuroscience.

\subsection{Calculating the Infinitesimal Phase Response Curve of Piecewise Smooth Systems}
My masters work is on the infinitesimal phase response curves (iPRCs) of planar piecewise smooth dynamical systems \cite{park2013infinitesimal}. The iPRC is often a necessary tool for predicting entrainment and synchronization of weakly coupled or weakly forced oscillators. In piecewise smooth systems, the iPRC may be discontinuous due to the piecewise defined vector field. My masters thesis shows how to compute the exact size of these discontinuities, provided that solutions are continuous and transversely cross boundaries between the piecewise defined vector fields.

\subsection{Reducing Dynamics Through a Separation of Timescales}
My doctoral work is on smooth dynamical systems, in particular the reduction of unwieldy high-dimensional systems to low-dimensional systems that are more amenable to a classic dynamical systems analysis. In the systems I have considered with my doctoral advisor Bard Ermentrout, the reduction is possible due to a natural separation of timescales.

\subsubsection{Synchronization of Weakly Coupled Cortical Neurons Modulated by a Slowly Varying Concentration of Acetylcholine}
The first project on this theme is on the synchronization of conductance-based neural models modulated by a slowly varying level of acetylcholine. In this system, the period of the neural spikes are orders of magnitude faster than the varying concentration of acetylcholine. This separation of timescales allows for a phase reduction and a standard weak coupling analysis \cite{park2016weakly}.

\subsubsection{Spatio-Temporal Dynamics of a Bump Solution of a Neural Field Model}
In my second project, I analyze a particular neural field model, first studied in \cite{pinto_ermentrout_2001_siam}, with weak and slow adaptation. Generally, neural field models are capable of producing a vast number of spatio-temporal patterns \cite{breakspear2017dynamic}, but this model produces a single bump solution that exhibits translational movement depending on the strength of a weak input current and the weak adaptation variable. We reduce the infinite dimensional partial integro-differential equation to a system of two scalar delay integro-differential equations, then analyze the many bifurcations in the movement of the bump solution on the ring and torus. We do not use non-smooth assumptions like the high-gain limit of the firing rate function. This paper is close to submission.


\section{Recent Work}
My doctoral work continues in the form of collaborations, and will conclude on a third paper. My masters work also continues in the form of an $n$-dimensional generalization of the planar result of my masters thesis.

\subsection{The Connection Between the Mean-Field Description and Spiking Neurons}
In the third and final paper of my Doctorate work, we seek to rigorously derive the connection between mean field neural networks and neural activity at the spiking level. In particular, we consider a system of synaptically coupled oscillators that generate a limit cycle oscillation at the mean field level in the synaptic variables. The synchronization properties at the spiking level are not well understood.

Our strong assumption is that the synapses of the network must be sufficiently slow. In this case, we can rephrase the problem as a weak coupling problem \cite{rubinrubin}. Although great strides are made in \cite{rubinrubin} and \cite{park2016weakly} to help connect the spiking description to the mean field description, the story is still incomplete. In \cite{rubinrubin}, the underlying slowly varying activity was kept constant to allow for a standard phase reduction. The results in \cite{park2016weakly} partially address this issue by deriving a phase reduction technique when there is an exogenous slowly varying parameter that is independent of the fast timescale activity. However, the slow dynamics of this third project are generated as part of the dynamical system, and we have yet to derive a proper phase reduction in this case.

\subsection{General Weakly Coupled Piecewise Smooth Oscillators}
In addition to these primary studies, I have other collaborations. For instance, I have continued to build on my masters work in continued collaborations with my masters thesis advisor Peter J. Thomas (with colleagues Hillel J. Chiel and Kendrick M. Shaw). We have written a manuscript for an $n$-dimensional generalization of this result and have started to apply the result to a weakly coupled piecewise smooth systems \cite{park2016infinitesimal}. Most existing literature on coupling of piecewise smooth systems are often restricted to one or a combination of linear coupling, planar systems, and piecewise linear vector fields \cite{coombes2016synchrony,izhikevich2000phase,coombes2012nonsmooth}. Although we do require solutions to be continuous, our weak coupling analysis applies to piecewise smooth systems of arbitrary dimension with nonlinear, heterogeneous coupling. This work has been submitted.

\subsection{A New Way to Compute Isochrons}
In another collaboration, I am working with another graduate student named Benjamin G. Letson to construct a numerical scheme for the computation of isochrons of limit cycles without using \texttt{AUTO}. The underlying analysis involves a change of coordinates to a moving frame with axes that are perpendicular and parallel to the underlying limit cycle solution. In this new frame, isochrons are relatively easy to visualize in principle. There are many numerical details yet to be ironed out.

\section{Future Directions}
Each of this work has several potential, fruitful directions. The extension of weak coupling theory to piecewise smooth oscillators opens a path to a rigorous re-rerivation of the classic weak coupling results to an extremely rich class of dynamical systems. I am excited to see the how far one could extend the classic results on weak coupling.

In my second paper for my doctorate, we assume a time-invariant input current. If the input current is allowed to vary, there is potential for an entirely different set of phenomena in addition to what is already a rich set of behaviors.

As implicitly mentioned above, my doctorate thesis depends strongly on a weak and slow assumption. These assumptions are rather general and could be found in many other systems. For example, traveling waves with some sufficiently slow variables could be amenable to a multiple timescale analysis, where the reduced dynamics could describe the velocity of the wave and the direction of the wave front on a two dimensional domain. Any neural field model that produces a simple, structurally coherent pattern with some weak and slow dynamics are good candidates for this reduction analysis.


\bibliographystyle{plain}
\bibliography{../youngmin-bard/bibliography,../youngmin-bard/ymp,../youngmin-bard/bio,../youngmin-bard/neuralfield}


\end{document}
