
\documentclass[a4paper,11pt]{letter}
\usepackage[utf8]{inputenc}

% extra packages
\usepackage{amsmath, amssymb}
\usepackage{graphicx}
\usepackage{color}
\usepackage{hyperref}

%\setlength{\parindent}{0pt}
% 
% \signature{Youngmin Park}
% \address{936 Lilac St. \\ Pittsburgh \\ PA 15213}

\begin{document}

\begin{center}
\Large \textbf{Teaching Statement}

\Large Youngmin Park
\end{center}

My teaching is fundamentally based on the philosophy that learning and understanding come with practice and context. To this end I provide a mixture of challenges of varying difficulty in the form of assignments and in-class exercises to get students to learn the basic concepts and tools. Then my lectures complement the textbook by explaining missing steps and important applications that may be glossed over in favor of technical details

The problems I choose for assignments and in-class exercises vary in difficulty and scope. For assignments, I typically choose a broad range, from rote "plug-and-chug" applications of formulas, to more involved word problems and proofs. The proof-based problems are a personal favorite because they tend to provide an upper bound for the mathematical ability of the class, allowing me to adjust my teaching style accordingly. For example, a class that struggles largely on proof-based problems suggests that I should focus much more on concrete applications (assuming proofs are not an integral part of the course), while a class that finds proof-based problems easy might benefit from a more abstract approach.

I find that in-class exercises are absolutely crucial to both my students' understanding of the subject, and my adjustment to my students' learning. I assign students to work in pairs on straightforward problems and encourage discussion and collaboration between pairs. This active discussion leads to a mutually beneficial give-and-take for the students. The students are able to teach each other in their own words, which allows for a much better consolidation of new information than listening to the lecturer. From my perspective, this discussion is invaluable because it produces questions as students get stuck. The questions often point to a weakness in either my teaching or the students' ability. From here, I provide more of the same problem in a subsequent lecture, assignment, or in-class exercise.

My lectures often closely follow the textbook, but I personally find that learning mathematics without knowing where it came from or why it's used can be dangerously disengaging. Thus, I mention history and applications where appropriate. For example, in my linear algebra classes, I take time to mention that the determinant (today learned as a property of matrices) was known long before matrices in the form of systems of equations used by ancient Chinese mathematicians. Matrices as we know them today were formalized many centuries later with effort spanning at least one mathematical career. Indeed, we see the payoff of this effort in the fundamental and ubiquitous usage of matrices and arrays in mathematics, science, engineering, and computer science.

In other lectures, I mention the many uses of eigenvalues in biology, physics, and chemistry. In particular, imaginary eigenvalues with a real component that changes from negative to positive (or vice-versa) plays a role in an incredible number of spatio-temporal dynamics including examples at a microscopic scale with the formation of repetitive neural spiking with sufficient input stimulus, to examples at a macroscopic scale with the formation of oscillating cortical waves observed during epileptic seizures.

These are just a few examples of the context I bring to my students. Combined with a rigorous regimen of mathematical drills, I optimize the consolidation of new information for my students.
 
 
\end{document}
