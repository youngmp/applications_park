\documentclass[a4paper,11pt]{article}
\usepackage[utf8]{inputenc}

% extra packages
\usepackage{amsmath, amssymb}
\usepackage{graphicx}
\usepackage{color}
\usepackage{hyperref}

%\setlength{\parindent}{0pt}

\begin{document}

\begin{center}
\Large \textbf{Future Research Interests and Goals}

\Large Youngmin Park
\end{center}

In the immediate future, I plan to complete my current doctoral work, and an $n$-dimensional generalization of the planar result of my masters thesis. In the long term, I am interested in learning the skills and mathematics necessary to address general problems that arise in neuroscience.

To conclude my Doctorate work, my advisor and I are seeking to rigorously derive the connection between mean field neural networks and neural activity at the spiking level. In particular, we consider a system of synaptically coupled oscillators (heterogeneous vector fields and possibly asymmetric coupling) that generate a slow timescale limit cycle oscillation at the mean field level of the synaptic variables. Our strong assumption is that the synapses of the network must be sufficiently slow. In this case, we can rephrase the problem as a weak coupling problem \cite{rubinrubin} where a standard phase reduction may be possible.

As this summary demonstrates, I have experience in working at the microscopic small-network scale, and mesoscopic continuum limit. I believe my contributions in applied math are sigificant and personally fulfilling, but I am strongly interested in transitioning to research more closely related to real data, as well as addressing problems more in the realm of theoretical neuroscience.


\bibliographystyle{plain}
\bibliography{../../youngmin-bard/bibliography,../../youngmin-bard/ymp,../../youngmin-bard/bio,../../youngmin-bard/neuralfield}


\end{document}
