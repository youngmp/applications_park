\documentclass[a4paper,11pt]{article}
\usepackage[utf8]{inputenc}

% extra packages
\usepackage{amsmath, amssymb}
\usepackage{graphicx}
\usepackage{color}
\usepackage{hyperref}

%\setlength{\parindent}{0pt}

\begin{document}

\begin{center}
\Large \textbf{Past Research Experience}

\Large Youngmin Park
\end{center}

\section{Background}
My research has largely focused on the study of deterministic, smooth and non-smooth dynamical systems, especially systems that arise in mathematical neuroscience.

My masters work is on the infinitesimal phase response curves (iPRCs) of planar piecewise smooth dynamical systems \cite{park2013infinitesimal}. The iPRC is often a necessary tool for predicting entrainment and synchronization of weakly coupled or weakly forced oscillators. In piecewise smooth systems, the iPRC may be discontinuous due to the piecewise defined vector field. My masters thesis shows how to compute the exact size of these discontinuities, provided that solutions are continuous and transversely cross boundaries between the piecewise defined vector fields.

My doctoral work is on smooth dynamical systems, in particular the reduction of unwieldy high-dimensional systems to low-dimensional systems that are more amenable to a classic dynamical systems analysis. In the systems I have considered with my doctoral advisor Bard Ermentrout, the reduction is possible due to a natural separation of timescales.

The first project on this theme is on the synchronization of conductance-based neural models modulated by a slowly varying level of acetylcholine. In this system, the period of the neural spikes are orders of magnitude faster than the varying concentration of acetylcholine. This separation of timescales allows for a phase reduction and a standard weak coupling analysis \cite{park2016weakly}.

In my second project, I analyze a particular neural field model, first studied in \cite{pinto_ermentrout_2001_siam}, with weak and slow adaptation. Generally, neural field models are capable of producing a vast number of spatio-temporal patterns \cite{breakspear2017dynamic}, but this model produces a single bump solution that exhibits translational movement depending on the strength of a weak input current and the weak adaptation variable. We reduce the infinite dimensional partial integro-differential equation to a system of two scalar delay integro-differential equations, then analyze the many bifurcations in the movement of the bump solution on the ring and torus. We do not use non-smooth assumptions like the high-gain limit of the firing rate function. This paper has been submitted to SIADS.


\bibliographystyle{plain}
\bibliography{../../youngmin-bard/bibliography,../../youngmin-bard/ymp,../../youngmin-bard/bio,../../youngmin-bard/neuralfield}


\end{document}
