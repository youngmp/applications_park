\documentclass[a4paper,12pt]{article}
\usepackage[utf8]{inputenc}

% extra packages
\usepackage{amsmath, amssymb}
% \usepackage[round]{natbib}
\usepackage{graphicx}
\usepackage{caption}
\usepackage{subcaption}
\usepackage{color}
\usepackage{hyperref}
\usepackage[normalem]{ulem}
\usepackage{geometry}
\usepackage{microtype}
% \usepackage{breakcites}
%\geometry{margin=.5in}

% to use breakcites in linux, run
% sudo apt-get install texlive-bibtex-extra

%\setlength{\parindent}{0pt}

% Bard's shorthand commands
\newcommand{\crefrangeconjunction}{--}
\newcommand{\x}{\mathbf{x}}
\newcommand{\y}{\mathbf{y}}
\newcommand{\z}{\mathbf{z}}
\newcommand{\F}{\mathbf{F}}
\newcommand{\q}{\boldsymbol{\theta}}
\newcommand{\io}{\int_\Omega}
\newcommand{\ve}{\varepsilon}
\newcommand{\pa}{\partial}
\newcommand{\R}{\mathbb{R}}



%opening
\title{Literature Review for Maria Geffen Lab}
\author{Youngmin Park}

\begin{document}

\maketitle
\tableofcontents

The purpose of this document is to detail possible computational projects related to auditory coding.

\section{Biophysical Mechanism for Gain Control}
Many of Maria Geffen's papers involve using a linear-non-linear model (LN model) to fit firing rate data as a function of a convolution of the neural receptive field with the input signal (see for example \cite{natan2016gain}). This type of model fitting is of interest because gain control of firing rates appears to be a hallmark of many neural coding problems.

In particular, gain control depends on the degree of spatiotemporal contrast of the input stimulus. In \cite{rabinowitz2011contrast}, the authors find that if the standard deviation in stimulus spatiotemporal contrast, the neurons have a relatively high gain, and when the standard deviation of the input stimulus is high, the neurons have a relatively low gain.

The biophyical mechanism for this gain control is unknown, although several hypotheses exist. The authors of \cite{rabinowitz2011contrast} cite two papers. The first paper uses a modified LN model. The start with a classic LN model, where the receptive field is convolved with the local stimulus contrast in time, then this scalar is input into a rectifying nonlinearity to estimate the firing rate of the population. The modification before and after the rectifying nonlinearty: a simple ciruit is added before the rectification, which include a conductance $g$. This conductance depends on a pool of additional neurons where the aggregate behavior affects the conductance $g$ depending on an additional parameter $k$. Aspects of these choices are for mathematical convenience with no physiological basis. However, the authors are successful at producing a shunting inhibition \cite{carandini1997linearity}.

The second paper suggests that given synaptic balance, the rate of inhibitory and excitatory inputs contributes to gain control \cite{abbott2005drivers}. They demonstrate using an integrate and fire model the type of divisive change in firing rate as a function of input current that are observed in experiments.

Naively, the second paper may be a better candidate for modeling, as the biophysical basis is more realistic. In addition, balanced networks are an active area of mathematical research with a wealth of resources available to analyze such networks. Combined with \cite{natan2017cortical}, there are enough raw materials to begin building a model to describe gain control.

\section{Mean Field Description of the Auditory Cortex}


\bibliographystyle{plain} 
\bibliography{audio}



\end{document}


