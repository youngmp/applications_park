\documentclass[a4paper,11pt]{article}
\usepackage[utf8]{inputenc}

% extra packages
% \usepackage{amsrefs}
% \usepackage[autocite=inline,labelalpha=true]{biblatex}

%\usepackage[top=1in, left=1in, right=1in, bottom=1in]{geometry}
\usepackage[top=1.5in, left=1.45in, right=1.45in, bottom=1.5in]{geometry}
\usepackage{wrapfig}
\usepackage{amsmath, amssymb}
\usepackage{graphicx}
\usepackage{color}
\usepackage{hyperref}
\usepackage{siunitx}

\DeclareUnicodeCharacter{03F5}{\ensuremath{\epsilon}}

\newcommand{\pa}{\partial}
\newcommand{\ve}{\varepsilon}
\newcommand{\h}{\mathcal{H}}

%\setlength{\parindent}{0pt}

\begin{document}
	
	\begin{center}
		\Large \textbf{Research Statement}
		
		\Large Youngmin Park
	\end{center}

    \begin{itemize}
        \item \textbf{Mathematical themes}: Stochastic and deterministic dynamical systems. Oscillators, separation of timescales, bifurcation theory, master equation.
        \item\textbf{Biological themes}: Biological neural networks. Cortical function, molecular motor transport, gene expression.
    \end{itemize}
	
	\section{Research Overview}
	
	I develop dimension-reduction methods for neural network models using stochastic and deterministic \textbf{dynamical systems theory}. In turn, I use these methods to understand the function and maintenance of biological and chemical networks within the context of \textbf{neuroscience}. My work is highly \textbf{interdisciplinary with excellent funding potential}. My publication record demonstrates my ability to produce high-impact work with researchers of diverse backgrounds, including neuroscientists \cite{park2020circuit,shaw2012phase}, engineers \cite{ermentrout2019recent,park2021high}, mathematical neuroscientists \cite{park2016weakly,park2018infinitesimal,park2018multiple,park2018scalar}, and fluid dynamicists \cite{park2020dynamics,park2021coarse,fai2020global}. My research program includes the following sub-directions:
	
	\begin{itemize}
		\item \textbf{Neural Oscillators}. In \cite{park2021high} I introduce generalizations of weakly coupled oscillator theory to strong coupling. This work enables engineers to design optimally controlled systems related to rhythms such as heart beats, sleep cycles, and central pattern generators.
		\item \textbf{Neural Maintenance}. In \cite{park2021coarse,park2020dynamics} I analyze the dynamics of molecular motors, which are necessary for cell function in all animals. Research includes developing a master equation of motor states coupled to a PDE describing the underlying motor positions and the exploration of discrepancies in stochastic and deterministic models related to Keizer's paradox.
		\item \textbf{Auditory Cortex}. In \cite{park2020circuit} I introduce an idealized model of the auditory cortex. This model unifies disparate optogenetics results in the literature and will help neuroscientists ask more targeted questions regarding auditory processing.
		\item \textbf{Undergraduate Research}. My work is accessible to a broad spectrum of skills and backgrounds in STEM (although I strongly encourage students from other backgrounds to participate in my research). My goal is to equip students with programming and scientific literacy skills, which they can use to enhance their lives and careers.
	\end{itemize}
	%I conclude with a comprehensive research plan in Section \ref{sec:research}.
	
	\newpage

	\bibliographystyle{plain}
	\bibliography{refs.bib,../youngmin-bard/bio,../youngmin-bard/neuralfield,../youngmin-bard/math,../youngmin-bard/phase,../youngmin-bard/computation,../youngmin-bard/cortex,../thomas-youngmin/notes/spines.bib,../thomas-youngmin/notes/vesicles.bib,../thomas-youngmin/notes/noise.bib,../thomas-youngmin/notes/motors.bib}
	
	
\end{document}
