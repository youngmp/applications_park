\documentclass[a4paper,11pt]{article}
\usepackage[utf8]{inputenc}

\usepackage{longtable}
% extra packages
\usepackage[top=1in, left=1in, right=1in, bottom=1in]{geometry}
\usepackage{amsmath, amssymb}
\usepackage{graphicx}
\usepackage{color}
\usepackage{hyperref}

%\setlength{\parindent}{0pt}

\begin{document}
	
	\begin{center}
		\Large \textbf{Diversity Statement}
		
		\Large Youngmin Park
	\end{center}
	
	I am a first-generation native South Korean who grew up all over the world (including Swaziland, South Korea, and the Philippines), but by and large grew up in poverty in the United States. I also grew up with undiagnosed mental health disorders, constantly being told to "just pay attention" or "just be happy". These experiences have taught me the importance of diversity and accepting neurodivergence. Just as organized groups actively and honorably fight for better handicap access. Likewise, I advocate for 
	
	While I have lived in the US since adolescence, this country is incredibly unique and I have continuously adjusted over my lifetime to fit into its culture. This experience has helped me understand the difficulty of adjusting to a new life in the US while attending a competitive program. We must actively help students who struggle in this way for the good of the nation's economy and security \cite{jones2018call}.
	
	Looking back on my childhood, I am struck that none of my friends from poor families earned a college degree, while all friends from well-off families went on to college and beyond. Anyone with even a hint of mental health disorders had no opportunities. Even though all of us went to the same, reasonably well-funded elementary, middle, and high schools, we took very different paths. My friends who did not attend college are brilliant people who would have thrived in that environment and could have greatly advanced their careers. They are people who could have been promoted to positions with greater sway over greater numbers of people, allowing them to guide policies and make decisions benefiting their community, city, state, and country. The world pushed them away from their full potential. Indeed, my anecdotal experience aligns with known demographic differences \cite{jones2018call} due in part to a lack of student engagement \cite{kokkelenberg2010succeeds,savaria2017critical}.
	
	I have taken steps to fight against this unfairness from the moment I started teaching as a graduate student at U Pitt through my postdoctoral teaching at Brandeis University and the University of Manitoba. My teaching style takes elements from expeditionary learning, inquiry-based learning, and differentiated instruction. The active component of my teaching is important because it has been shown to improve student performance in underrepresented groups in STEM \cite{theobald2020active}, while differentiated instruction allows me to tailor my teaching to the individual. \textit{The bottom line is that I have directly helped many students of underrepresented demographics with weaker backgrounds in mathematics to succeed in my courses by providing additional resources, encouraging them to meet with tutors, and meeting with them regularly outside of class and office hours.} My actions speak for themselves.
	
	I actively work to see a world where every student has access to teachers who genuinely want their students learn and succeed, based on the tenets of \textbf{diversity, equity, and inclusion}. To this end I have taught as a guest lecturer in science for underrepresented Bangladeshi children at Moder Patshala and the Free Library of Pennsylvania. At Brandeis, I volunteered to give neuroscience lectures on mental health at Waltham High School. At Manitoba, I will engage with ongoing efforts for education in the native population. In the future, I will continue to seek opportunities in science outreach and join organizations advocating for underrepresented groups, including the Society for Advancement of Chicanos/Hispanics and Native Americans in Science (SACNAS), and the Association for Women in Mathematics (AWM). \textbf{All people deserve the best from their teachers and I will not stop working to make this hope a reality}.
	
	%Racism causes results in preconceived notions formed by exaggerated circumstances such as those that occur in popular media. Every day I work to understand that each individual I meet is no more than my interactions with them, and that I should form my own opinions based on our mutual experiences. This philosophy naturally transfers to my teaching, where I evaluate a student's needs based directly on my interactions with them, as opposed to what I have experienced with people of a similar socio-economic status. It is a lifelong challenge that I wholeheartedly accept. Some students require more mentoring, while others recognize patterns very quickly. \textit{By adjusting my teaching based on the individual, I become a maximally effective teacher}. I help all of my students to the best of my abilities regardless of race, gender, background, and disability.
	
	\newpage
	\bibliographystyle{plain}
	\bibliography{edi}
	
\end{document}
