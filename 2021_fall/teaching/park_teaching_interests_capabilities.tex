\documentclass[a4paper,11pt]{article}
\usepackage[utf8]{inputenc}

\usepackage[usenames,dvipsnames]{xcolor}
\usepackage{longtable}
% extra packages
\usepackage[top=1in, left=1in, right=1in, bottom=1in]{geometry}
\usepackage{amsmath, amssymb}
\usepackage{graphicx}
\usepackage{color}
\usepackage{hyperref}
\usepackage{colortbl}

%\setlength{\parindent}{0pt}

\begin{document}

\begin{center}
\Large \textbf{Teaching Statement}

\Large Youngmin Park
\end{center}



\section{Teaching Philosophy}

\textit{I teach based on the principle that practice and context drive robust learning}. In practice, I do not use a single teaching style, but draw elements from expeditionary learning, inquiry-based learning, flipped classrooms, direct instruction, and differentiated instruction. My flexibility allows me to best address the needs of different classrooms and individual students. I implement these methods in the following ways:

\begin{enumerate}
	\item \textbf{Expeditionary learning} is implemented in the form of student presentations, typically with more experienced undergraduates, in classes like linear algebra and differential equations. Students are encouraged to pick from a list of topics or formulate their own topic. For example, my Summer 2016 class was given the list:
	\begin{itemize}
		\item Write a program to reproduce the bifurcation diagram of the logistic map.
		\item Write a program to reproduce the Fourier series of classic functions: square wave, sawtooth. Demonstrate improved accuracy with more Fourier terms.
		\item Write a program for Forward Euler, 2nd order Runge-Kutta, and 4th order Runge-Kutta. Compare accuracy.
		\item Series solutions of differential equations: discuss several examples of this solution method.
		\item The method of Lyapunov. Present the theorem and show examples of the method.
		\item Matrix exponentials. Show how to solve linear ODEs using matrix exponentials.
	\end{itemize}
	While I provide close guidance in preparing the presentation and research, students are given great independence in exploring the depth and breadth of their topic. This approach includes elements of \textbf{personalized teaching}, where students are allowed to master topics at their own pace, and are given additional guidance as needed. My students not only enjoyed this process, but found that it helped them understand detailed and big-picture concepts.
	
	\item \textbf{Inquiry-based Learning} is implemented in every lecture by encouraging both myself and my students to ask questions about the work. Why is it important to know derivatives? What is the use of the matrix null space? This process of questioning has led me to become a better teacher by helping students understand why and how some concepts were introduced. For example, based on my students' questions, I felt that my explanation of nullspace was insufficient, learning me to write a Python script demonstrating how matrices deform shapes. The visualization better explained the concepts of nullspace, range, and linear transformations. As another example, also based on my students' questions, I explained how matrix notation took the better part of a mathematician's career to standardize, and the notation is relatively young compared to the millennia-long history of linear equations. Through inquiry-based learning, my students have a voice in how they learn.
	
	\item \textbf{Flipped classrooms} are implemented on days dedicated to problem-solving. Near the beginning of the semester, these problems are straightforward applications of formulas, definitions, and theorems, which are crucial for a baseline understanding of advanced concepts. This stage often appears as \textbf{direct instruction}. Students are assigned a series of reading sections or problems prior to class, so they are prepared to learn how to solve problems. As the semester progresses, I introduce increasingly challenging problems that push the students to connect multiple concepts, and perhaps, to generate their own. To assist in problem-solving, I assign students to work in pairs and encourage discussion, which leads to a mutually beneficial give-and-take: as the students encounter difficulties, they often overcome them autonomously while teaching each other in their own words. If they are unable to overcome a hurdle, they ask for help, which makes me understand the specific challenges faced by my students. Through this process I tailor my teaching to adjust to individual students. 
	
	\item \textbf{Differentiated instruction} is implemented at all times. While my default teaching style is geared towards a broad spectrum of abilities, it is not always the case that students will excel in such a classroom environment. I draw on my experience as a student who struggled in classroom environments despite being able to learn in other contexts, so I am sensitive to students who struggle in class and especially mindful of those who seek help outside of class. I have found that one-on-one meetings are excellent for tailoring my teaching to the individual in a way that would not be possible in a lecture. For example, one student struggled to understand fundamental proof concepts while solving a series of homework problems. Over several meetings, I explained these fundamental concepts until the student understood them completely. The student struggled but persevered and ultimately excelled in the course.
\end{enumerate}

To conclude, I remark that my teaching efforts are always \textbf{independent of a student's race, gender, or socio-economic background}. I actively work to see a world where every student has access to teachers who genuinely want their students learn and succeed, based on the tenets of \textbf{diversity, equity, and inclusion}. To this end I have taught as a guest lecturer in science for underrepresented Bangladeshi children at Moder Patshala and the Free Library of Pennsylvania. At Brandeis, I volunteered in science outreach to give neuroscience lectures at Waltham High School. In the future, I will continue to seek opportunities in science outreach and join organizations advocating for underrepresented groups, including the Society for Advancement of Chicanos/Hispanics and Native Americans in Science (SACNAS), and the Association for Women in Mathematics (AWM). \textbf{All people deserve the best from their teachers and I will not stop working to make this hope a reality}.

\newpage

\section{Teaching Experience}

\noindent\rule{15cm}{0.4pt}

\begin{tabular}{p{0.11\linewidth}p{0.4\linewidth}p{.3\linewidth}}
	\textit{Type} & \textit{Class} & \textit{Term(s)}
\end{tabular}

\noindent\rule{15cm}{0.4pt}

\textbf{University of Manitoba (2021--2023)}

\begin{tabular}{p{0.11\linewidth}p{0.4\linewidth}p{.3\linewidth}}
	{Lecture} & Partial Differential Equations & Spring 2022\\
    {Lecture} & Ordinary Differential Equations & Fall 2021
\end{tabular}


\noindent\rule{15cm}{0.4pt}

\textbf{Brandeis (2019--2021)}

\begin{tabular}{p{0.11\linewidth}p{0.4\linewidth}p{.3\linewidth}}
    {Lecture} & Calculus 3 & Spring 2021\\
    {Lecture} & Linear Algebra & Spring 2020
\end{tabular}

\noindent\rule{15cm}{0.4pt}

\textbf{University of Pittsburgh (2013--2018)}

\begin{tabular}{p{0.11\linewidth}p{0.4\linewidth}p{.3\linewidth}}
	 Lecture & Differential Equations (3 sections) & Summers, 2014--2017\\
	 & Linear Algebra & Summer 2015 \\
	 & Discrete Math & Spring 2015 \\
	 Recitation & Computational Neuroscience & Summers, 2014--2017 \\
	  & Business Calculus (6 sections) & Fall/Spring 2013/16\\
	  & Calculus 1, 2, 3 (6 sections) & Fall/Spring 2014--2016\\
	 Grading & Differential Equations (10 sections) & Fall/Spring 2013--2017  \\
	 & Complex Variables and Applications & Spring 2017 \\
	 & Linear Algebra (2 sections) & Spring 2016
\end{tabular}

\noindent\rule{15cm}{0.4pt}

\textbf{Oberlin (2013)}

\begin{tabular}{p{0.11\linewidth}p{0.4\linewidth}p{.2\linewidth}}
	 Assistant & Computational Neuroscience  & Winter 2013 
\end{tabular}

\noindent\rule{15cm}{0.4pt}

\textbf{Case Western Reserve University (2012)}

\begin{tabular}{p{0.11\linewidth}p{0.4\linewidth}p{.2\linewidth}}
	Assistant & Calculus 3  & Spring 2012
\end{tabular}

%\vspace{1cm}

%My teaching experience spans nearly ten years at five institutions. My teaching style has consistently led to strong teaching evaluations and a shortlist for the Elizabeth Baranger teaching award, which serves to recognize and reward outstanding teaching by graduate students at the University of Pittsburgh.

%When teaching as a lecturer, I independently designed each course and prepared all materials including lectures, quizzes, tests. I taught classes of varying sizes, ranging from 9 to 50 students. As a course assistant, grader, or recitation leader, I coordinated with the instructor to best evaluate students through quizzes and homework assignments. At the University of Pittsburgh, I became closely acquainted with Lon Capa to automate homework assignments while teaching numerous courses in the calculus sequence. All recitations were supplemented by office hours and additional appointments as needed.


\section{Summary of Selected Teaching Evaluations}
%\begin{tabular}{p{0.11\linewidth}p{0.4\linewidth}p{.3\linewidth}}
\begin{tabular}{lllll}
	Term & School & Course & Type & \textbf{Evaluation}\\
	\hline
	Spring 2021 &Brandeis & Calculus 3 & Lecture &  \textbf{4.53}/5\\
	Spring 2020 &Brandeis & Linear Algebra & Lecture &  \textbf{4.7}/5\\
	Summer 2017 &U Pitt & Differential Equations & Lecture &  \textbf{4.86}/5\\
	Summer 2016 &U Pitt & Differential Equations & Lecture &  \textbf{4.4}/5\\
	Summer 2015 &U Pitt & Linear Algebra & Lecture &  \textbf{4.24}/5\\
	Fall 2014 &U Pitt & Calculus 1 & Recitation &  \textbf{4.5}/5\\
\end{tabular}

 
\end{document}
