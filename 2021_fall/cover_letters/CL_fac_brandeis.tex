%%%%%%%%%%%%%%%%%%%%%%%%%%%%%%%%%%%%%%%%%%%%%%%%%%%%%%%%
%%%%%%%%%%%------RMIT Letterhead------------%%%%%%%%%%%%
%%%%%%%%%%%%%%%%%%%%%%%%%%%%%%%%%%%%%%%%%%%%%%%%%%%%%%%%
\documentclass[11pt,a4paper]{letter}

%\usepackage[english]{babel}
%\usepackage[utf8x]{inputenc}
%\usepackage[utf8]{inputenc}

%\usepackage[T1]{fontenc}

% you may specify the font size (10pt, 11pt and 12pt) and paper size (letterpaper, a4paper, etc.)
\usepackage{microtype}
\usepackage{graphicx}
\usepackage{gfsdidot}
\usepackage[T1]{fontenc}

\documentclass[11pt,a4paper]{letter}

\usepackage{microtype}
\usepackage{graphicx}
\usepackage{gfsdidot}
\usepackage[T1]{fontenc}
\usepackage{geometry}
\geometry{
	a4paper,
	left=20mm,
	right=20mm,
	headheight=.0cm,
	top=.0cm,
	bottom=0cm,
	footskip=0cm
}




%----------------------------------------------------------------------------------------
%	SENDER INFORMATION
%----------------------------------------------------------------------------------------

\def\Who{Youngmin Park, Ph.D.}
\def\What{, PIMS Postdoctoral Fellow}
\def\Where{Department of Mathematics\\University of Manitoba}
\def\addy{420 Machray Hall 186 Dysart Rd}
\def\CityZip{Winnipeg, MB R3T 2N2 Canada\\\today}
\def\Email{E-mail: ypark@brandeis.edu}
\def\TEL{Phone: (412) 805-0283}
\def\URL{youngmp.github.io}
%----------------------------------------------------------------------------------------
%	HEADER AND FROM ADDRESS STRUCTURE
%----------------------------------------------------------------------------------------
%%%%
%\vspace{-2in}
\def\address{
	\raisebox{0in}[0pt][0pt]{\includegraphics[width=1.2in]{logo_manitoba.png}} % Include the logo of your institution
	\hspace{-3.31in} % Position of the institution logo, increase to move left, decrease to move right
	\vskip -.35in~\\ % Position of the text in relation to the institution logo, increase to move down, decrease to move up
	%\makebox[0ex][r]{\bf \Who \What }\hspace{0.08in}
	~\\[-0.11in] % Reduce the whitespace above the horizontal rule
	\hspace{\fill}\parbox[t]{2.85in}{ % Create a box for your details underneath the horizontal rule on the right
		\footnotesize % Use a smaller font size for the details
		\Who \\ \em % Your name, all text after this will be italicized
		\Where\\ % Your department
		\addy\\ % Your address
		\CityZip\\ % Your city and zip code
		%\TEL\\ % Your phone number
		%\Email\\ % Your email address
		%\URL % Your URL
	}
	\hspace{-1in} % Horizontal position of this block, increase to move left, decrease to move right
	\vspace{-1.6in} % Move the letter content up for a more compact look
}


%----------------------------------------------------------------------------------------
%	TO ADDRESS STRUCTURE
%----------------------------------------------------------------------------------------


\def\opening#1{\thispagestyle{empty}
	%%% from address
	{\centering\address \vspace{.75in} \\
		%% Date
		\hspace*{\longindentation}\hspace*{\fill}\par} % remove \today to not display it
	{\raggedright \toname \\ \toaddress \par} % Print the to name and address
	
	\noindent #1 % Print the opening line
}


%----------------------------------------------------------------------------------------
%	SIGNATURE
%----------------------------------------------------------------------------------------

\signature{\Who \What}

\long\def\closing#1{
	\vspace{0.1in} % Some whitespace after the letter content and before the signature
	\noindent % Stop paragraph indentation
	#1 % Print the signature text
	
	\includegraphics[width=1in]{signature.png} % Whitespace between the signature text and your name%
	
	\fromsig}


%\long\def\closing#1{
%\vspace{0.1in} % Some whitespace after the letter content and before the signature
%\noindent % Stop paragraph indentation
%\hspace*{\longindentation} % Move the signature right
%\parbox{\indentedwidth}{\raggedright
%#1 % Print the signature text
%
%\includegraphics[width=1in]{signature.png} % Whitespace between the signature text and your name%
%
%\fromsig}}

%----------------------------------------------------------------------------------------
% Create a new command for the horizontal rule in the document which allows thickness specification
\makeatletter
\def\vhrulefill#1{\leavevmode\leaders\hrule\@height#1\hfill \kern\z@}
\makeatother



%----------------------------------------------------------------------------------------
%	PAPER MARGINS
%----------------------------------------------------------------------------------------

\textwidth 6in
\textheight 11in
%\oddsidemargin -.4in
\oddsidemargin 0in
\evensidemargin 0in
\topmargin -1.2in
\longindentation 0.25\textwidth
\parindent 0.2in


\begin{document}

%----------------------------------------------------------------------------------------
%	TO ADDRESS
%----------------------------------------------------------------------------------------
\def\School{Brandeis University}
\begin{letter}
{Hiring Committee\\
Department of Mathematics, MS 050\\
Brandeis University\\
415 South Street\\
Waltham, MA 02453}

%----------------------------------------------------------------------------------------
%	LETTER CONTENT
%----------------------------------------------------------------------------------------

\opening{Dear Members of the Hiring Committee,}

I am applying for the position of Tenure-Track Assistant Professor in Mathematics at \School. I hold a Ph.D. in mathematics from the University of Pittsburgh advised by G. Bard Ermentrout, and now hold a postdoctoral position at Brandeis University advised by Thomas G. Fai. I specialize in \textbf{deterministic and stochastic dynamical systems}, \textbf{bifurcation theory}, \textbf{perturbation methods}, and \textbf{non-local integro-differential equations}, with applications to chemistry, neuroscience, and cell physiology.

My doctoral research in dimension reduction resulted in winning the prestigious \textbf{Andrew Mellon Predoctoral Fellowship}, awarded to doctoral students of exceptional promise and ability. I was the first math-bio doctoral student at the University of Pittsburgh to receive this award. My research ability has matured through my postdocs: at the University of Pennsylvania, I collaborated closely with neuroscientists and published ground-breaking models to explain data produced by the world's leading auditory labs. At Brandeis, I have continued to develop my ability as an independent mathematician while publishing and collaborating in multiple fields, including coupled oscillators, molecular motor dynamics, and chemical networks.

\School~features strong researchers in mathematical biology, including \textbf{Thomas Fai} and \textbf{Jonathan Touboul}. My research experience in molecular motor transport and mathematical neuroscience will augment their work through collaborations and grants, and I will enhance the department's reputation through my independent research program. I use dynamical systems and bifurcation theory to solve problems of oscillator synchrony in chemical and biological networks and use machine learning to uncover network properties in biological systems.

I am also committed to providing high-quality and equal education for all my students, as evidenced by my teaching evaluations. My teaching portfolio boasts eight years of teaching at different capacities (lectures, recitations, grading), at different levels (calculus sequence, differential equations, linear algebra, and discrete math), and at different institutions (Oberlin College, Case Western Reserve University, University of Pittsburgh, and Brandeis University). As a doctoral student, my students shortlisted me for the \textbf{Elizabeth Baranger teaching award}, the most prestigious teaching award at the University of Pittsburgh. In terms of outreach, I have served as a guest lecturer for underprivileged Bangladeshi children at the Free Library of Philadelphia, and will serve as a a guest lecturer in science at Waltham High School.

%I strongly believe in the principle that learning is best accomplished with context and practice. My students do not learn in a vacuum, but instead learn concepts alongside the human triumph and defeat behind many concepts we take for granted today, such as matrix notation and Euclid's fifth postulate. Students at \School~will greatly benefit from this approach, and in turn, will drive me towards excellent instruction.

As part of my application I include a curriculum vitae, research statement, and teaching statement. Please request additional details as needed, and I look forward to our correspondence.

\closing{Sincerely,}
%----------------------------------------------------------------------------------------
\end{letter}
\end{document}