\documentclass[a4paper,11pt]{article}
\usepackage[utf8]{inputenc}

\usepackage{longtable}
% extra packages
\usepackage[top=1in, left=1in, right=1in, bottom=1in]{geometry}
\usepackage{amsmath, amssymb}
\usepackage{graphicx}
\usepackage{color}
\usepackage{hyperref}

%\setlength{\parindent}{0pt}

\begin{document}

\begin{center}
\Large \textbf{Teaching Statement}

\Large Youngmin Park
\end{center}

\section{Teaching Experience}

My teaching experience spans four years and three semesters per year. In five of these semesters, I taught as the lecturer for three different classes: differential equations, linear algebra, and discrete math. My teaching style has consistently led to strong teaching evaluations, and I was shortlisted for the Elizabeth Baranger teaching award, which serves to recognize and reward outstanding teaching by graduate students at the University of Pittsburgh.

\begin{longtable}{rlll}
\textbf{Year} & \textbf{Term} & \textbf{Type} & \textbf{Class}\\
2017 & Summer& \textbf{Lecture} &  Differential Equations (14 students)\\
     & Spring & Grading &  Differential Equations 1 (25 students, x2)\\
     & & Grading & Differential Equations 2 (25 students)\\
     & & Grading & Complex Variables and Applications (25 students)\\
     & & Recitation & Comput. Neurosci. (21 students)\\
2016 & Fall & Recitation & Business Calculus (20--24 students each, x3)\\
     & Summer & \textbf{Lecture} & Differential Equations (23 students)\\
     & Spring & Recitation & Calculus 3 (28 students)\\
     & & Grading & Ordinary Differential Equations 1 (25 students, x2)\\
2015 & Fall & Recitation & Calculus 1 (25 students)\\
 &  & Recitation & Calculus 2 (25 students)\\
 &  & Grading & Ordinary Differential Equations 1 (25 students)\\
 & Summer & \textbf{Lecture} & Matrices and Linear Algebra (27 students)\\
 & Spring & \textbf{Lecture} & Discrete Math. Structures (33 students)\\
 & & Grading & Matrices and Linear Algebra (25 students, x2)\\
2014 & Fall & Recitation & Calculus 1 (25 students each, x3)\\
 & Summer & \textbf{Lecture} & Differential Equations (9 students)\\
2013 & Fall & Recitation &  Business Calculus (23 students)\\
 & & Grading & Differential Equations (25 students, x2)\\
%\textsc{Winter} 2013 &  
\end{longtable}

% 
% \begin{itemize}
%  \item Lectures
%  \begin{itemize}
%  \item \textbf{Differential Equations}
%  \item \textbf{Linear Algebra}
%  \item \textbf{Discrete Math}
% \end{itemize}
% \item Recitations
% \begin{itemize}
%  \item \textbf{Single-variable Calculus} (Calculus 1)
%  \item \textbf{Multi-variable Calculus} (Calculus 2 and 3)
%  \item \textbf{Computational Neuroscience}
%  \end{itemize}
%  \item Grading 
%  \begin{itemize}
%  \item \textbf{Differential Equations}
%  \item \textbf{Multi-variable Calculus} (Calculus 2 and 3). Volumes, coordinate systems, vectors, planes, sequences, series, differential equations, partial derivatives, optimization.
%  \item \textbf{Computational Neuroscience}. Neural models, ordinary differential equations, stochastic calculus, information theory.
%  \end{itemize}
% \end{itemize}


As the lecturer, I independently designed each course and prepared all materials including lectures, quizzes, tests, and homework assignments. In addition to grading, my teaching duties included meeting students during office hours, making additional appointments as needed. Each semester the class varied in size, ranging from 9 students to as many as 33. 

Another substantial part of my teaching portfolio includes serving as a teaching assistant and leading recitations, where the main lectures were given by a professor. These recitations were for single- and multi-variable calculus classes. In a typical semester, I led three recitation sections per week, where I spent one hour per section teaching calculus concepts (in coordination with the lecturer), and spent another hour working with students in a computer lab. In the lab, students solved automatically-generated calculus problems (generated using Lon Capa), and I provided appropriate hints as they got stuck.

I also led recitations for a course in computational neuroscience. In these recitations, I answered students' questions, and wrote MATLAB scripts on-the-fly to demonstrate simple concepts behind neural models, such as the numerical integration of ordinary differential equations. I also served as the grader for this course, which consisted of 21 students. All recitations were supplemented by office hours and additional appointments as needed.

% The final portion of my teaching portfolio includes grading. In addition to grading as a lecturer and recitation leader, I was often assigned to be a dedicated grader for particular classes, such as introductory differential equations (a one-semester course for engineers), advanced undergraduate differential equations (a two-semester course for math majors), and complex variables. In this role, I acted as a recitation leader, but met with students outside of class to help them better understand the material. These meetings took place during office hours and appointments as needed.



\section{Teaching Philosophy}

\textit{My teaching is fundamentally based on the belief that learning and understanding come with practice and context}. To this end I provide challenges of varying difficulty in the form of assignments and in-class exercises to maximize exposure to the material.

% My lectures complement the textbook by covering missing steps and important applications that may be glossed over in textbooks in favor of technical details.

The first step of familiarizing students with the material is to present and assign sufficient rote problems. These problems are ``plug-and-chug'' applications of formulas, which I believe to be absolutely crucial. In any other profession, improvement is achieved through practice. Musicians play scales and athletes drill. No matter the field, mastery of the most basic skills provides the foundation for advanced study. Through rote study, my students learn the notation, build a foundational understanding, and familiarize themselves with the language on which I build their knowledge.

The second step is to challenge the students. While rote practice is crucial, it is far from a complete learning paradigm. It is important to push students to see the bigger picture and apply fundamental skills in more challenging contexts. Derivatives are straightforward and are important to know, but they are most useful for solving problems such as related rates. Integral rules are good to know, but they are extremely useful for calculating areas and volumes. To aid in my students' understanding, I provide many examples of these applications.

To aid in these two steps, I assign students to work in pairs on straightforward problems and encourage discussion. This active discussion leads to a mutually beneficial give-and-take:  as the students encounter difficulties, they ask each other questions. They often overcome these difficulties autonomously, teaching each other in their own words. \textit{These are the steps I take to provide students with a wealth of practice}.

Learning mathematics without knowing where it came from or when it is used can be dangerously disengaging. To hedge against disengagement, I briefly cover history and applications where appropriate. In my linear algebra classes, I explain that the determinant -- which today is learned as a property of matrices -- was known long before matrices existed. It was the ancient Chinese that discovered determinants, and their mathematicians used the determinant to great effect solving systems of linear equations. Matrices as we know them today were formalized many centuries later, and only with great effort spanning many decades. Indeed, we see the payoff of this effort in the fundamental and ubiquitous usage of matrices and arrays in mathematics, science, engineering, and computer science.

In other lectures, I mention the many uses of eigenvalues in biology, physics, and chemistry. In particular, imaginary eigenvalues with a real component that changes from negative to positive (or vice-versa) plays a role in an incredible number of spatio-temporal dynamics, such as the formation of oscillating cortical waves observed during epileptic seizures, and the formation of patterns on animal hide. \textit{These are the steps I take to provide students with a wealth of context}.

Finally, it is critical to note that adjusting my teaching style through experience is a high priority. As good pedagogical practices become known, I implement them where appropriate. Through this experience, I have found that a combination of rote practice, challenging applications, and context are effective teaching tools. However, I hold myself to high teaching standards, and believe that this process will be a long-long journey.

\section{Diversity in Teaching}
I am a native South Korean who has lived as a minority for most of his life. I grew up with wealth in poverty-stricken countries such as Swaziland and the Philippines, and spent the latter part of my youth in poverty in wealthy nations such as Korea and the United States. My experiences with racism, such as the portrayal of Asian-Americans in popular media, and the experiences with poverty and the struggles of low-income peoples has profoundly influenced my conduct in social contexts. In all social situations, ranging from casual to professional, I am intensely aware of how my interactions affect (and do not affect) other people. \textit{I am a strong supporter of Boston University's commitment to diversity and inclusion}.

In the professional setting, my conscious efforts at diversity and inclusion primarily occur during teaching. In my four years of teaching, I have had the pleasure of teaching hundreds of students, where classrooms often consist of a large variety of socio-economic and educational backgrounds. My goal, first and foremost, is to maintain that \textit{I will give absolute and equal care to all students}. To this end, I work hard to keep my speech patterns and mannerisms consistent between all students. This conscious effort is of great personal importance in order to avoid the demeaning and demoralizing effect of differing or preferential treatment.

As part of my goal for diversity and inclusion, I always assist students with additional needs. One student required larger font (13 points at least) due to problems with his vision, so for each exam, I created an additional version with an augmented font size. He, as well as many other students, bring notes from a staff psychiatrist confirming their need for additional testing time, and that their tests be administered outside the classroom at a special testing facility on campus. For each student and for each exam, I email this external testing facility the exam with additional details such as the additional time required, and allowed materials such as calculators and notes. In all cases, I allow myself to be available by phone in order to answer questions during the exam.

Racism results in preconceived notions formed by exaggerated circumstances such as those that occur in popular media. Every day I work to understand that each individual I meet is no more than my interactions with them, and that I should form my own opinions based on our mutual experiences. This philosophy naturally transfers to my teaching, where I evaluate a student's needs based directly on my interactions with them, as opposed to what I have experienced with people of a similar socio-economic status. It is a lifelong challenge that I wholeheartedly accept. Some students require more details, while others recognize patterns very quickly, and I am sensitive to the skill-level of the individual. \textit{By adjusting my teaching based on the individual, I become a maximally effective teacher}.

Finally, good teaching requires a deep commitment that extends beyond the classroom. Thus, I never constrain my teaching to office hours, and often answer questions through email and make additional appointments as needed. I make it a point that this feature is available to all of my students, and I stress that I will help them to the best of my abilities regardless of race, gender, background, and disability.
 
\end{document}
