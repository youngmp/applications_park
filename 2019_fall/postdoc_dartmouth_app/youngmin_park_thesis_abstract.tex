\documentclass[a4paper,11pt]{article}
\usepackage[utf8]{inputenc}

% extra packages
\usepackage{amsmath, amssymb}

\usepackage{graphicx}
\usepackage{color}
\usepackage{hyperref}
\usepackage{footmisc}
\usepackage{url}

%\setlength{\parindent}{0pt}

\begin{document}

\begin{center}
\Large \textbf{PhD Thesis Abstract}

\Large Youngmin Park
\end{center}

In general, reducing the dimensionality of a complex model is a natural first step to gaining insight into the system. In this dissertation, we reduce the dimensions of models at three different scales: first at the scale of microscopic single-neurons, second at the scale of macroscopic infinite neurons, and third at an in-between spatial scale of finite neural populations. Each model also exhibits a separation of timescales, making them amenable to the method of multiple timescales, which is the primary dimension-reduction tool of this dissertation. In the first case, the method of multiple timescales reduces the dynamics of two coupled n-dimensional neurons into one scalar differential equation representing the slow timescale phase-locking properties of the oscillators as a function of an exogenous slowly varying parameter. This result extends the classic theory of weakly coupled oscillators. In the second case, the method reduces the many spatio-temporal \yp{dynamics of} ``bump'' solutions of a neural field model into its scalar coordinates, which are much easier to analyze analytically. This result generalizes existing studies on neural field spatio-temporal dynamics to the case of a smooth firing rate function and general even kernel. In the third case, we reduce the dimension of the oscillators at the spiking level -- similar to the first case --  but with additional slowly varying synaptic variables. This result generalizes existing studies that use scalar oscillators and the Ott-Antonsen ansatz to reduce the dimensionality and determine the synchronization properties of large neural populations.

\end{document}