\documentclass[a4paper,11pt]{article}
\usepackage[utf8]{inputenc}

\usepackage[usenames,dvipsnames]{xcolor}
\usepackage{longtable}
% extra packages
\usepackage[top=1in, left=1in, right=1in, bottom=1in]{geometry}
\usepackage{amsmath, amssymb}
\usepackage{graphicx}
\usepackage{color}
\usepackage{hyperref}
\usepackage{colortbl}

%\setlength{\parindent}{0pt}

\begin{document}
	
	\begin{center}
		\Large \textbf{Teaching Statement}
		
		\Large Youngmin Park
	\end{center}
	
	
	
	\section{Teaching Philosophy}
	
	\textit{I teach based on the principle that practice and context drive robust learning}. In practice, I do not use a single teaching style, but draw elements from expeditionary learning, inquiry-based learning, flipped classrooms, direct instruction, and differentiated instruction. My flexibility allows me to best address the needs of different classrooms and individual students. I implement these methods in the following ways:
	
	\begin{enumerate}
		\item \textbf{Expeditionary learning} is implemented in the form of student presentations, typically with more experienced undergraduates, in classes like linear algebra and differential equations. Students are encouraged to pick from a list of topics or formulate their own topic. While I provide close guidance in preparing the presentation and research, students are given great independence in exploring the depth and breadth of their topic. My students enjoy this process and find that it helps them understand core concepts.
		
		\item \textbf{Inquiry-based Learning} is implemented in every lecture by encouraging both myself and my students to ask questions about the work. Why is it important to know derivatives? What is the use of the matrix null space? This process of questioning has led me to become a better teacher by helping students understand why and how some concepts were introduced. For example, based on my students' questions, I felt that my explanation of nullspace was insufficient, learning me to write a Python script demonstrating how matrices deform shapes. The visualization better explained the concepts of nullspace, range, and linear transformations. %As another example, also based on my students' questions, I explained how matrix notation took the better part of a mathematician's career to standardize, and the notation is relatively young compared to the millennia-long history of linear equations. Through inquiry-based learning, my students have a voice in how they learn.
		
		\item \textbf{Flipped classrooms} are implemented on days dedicated to problem-solving. In the beginning of the semester, these problems are straightforward applications of formulas, definitions, and theorems, which are crucial for a baseline understanding of advanced concepts. This stage often appears as \textbf{direct instruction}. Students are assigned a series of reading sections or problems prior to class, so they are prepared to learn how to solve problems. As the semester progresses, I introduce increasingly challenging problems that push the students to connect multiple concepts, and perhaps, to generate their own. To assist in problem-solving, I assign students to work in pairs and encourage discussion, which leads to a mutually beneficial give-and-take: as the students encounter difficulties, they often overcome them autonomously while teaching each other in their own words. %If they are unable to overcome a hurdle, they ask for help, which makes me understand the specific challenges faced by my students. Through this process I tailor my teaching to adjust to individual students. 
		
		\item \textbf{Differentiated instruction} is implemented at all times. While my default teaching style is geared towards a broad spectrum of abilities, it is not always the case that students will excel in a classroom environment. In fact, I was a student who struggled in classroom environments despite being able to learn in other contexts. Therefore, I am sensitive to students who struggle and especially those who seek help during office hours. I have found that these one-on-one meetings are excellent for tailoring my teaching to the individual in a way that would not be possible in a lecture. %For example, I  had a student struggle to understand basic proof concepts while establishing an upper bound using the triangle inequality. Explaining the concepts took substantial time and effort, but the student ultimately excelled in the course. I do not hesitate to help my students in these moments.
	\end{enumerate}
	
	To conclude, I remark that my teaching efforts are always \textbf{independent of a student's race, gender, or socio-economic background}. I hope to see a world where every student has access to teachers who genuinely want their students learn and succeed, based on the tenets of diversity, equity, and inclusion. All people deserve the best from their teachers and I will not stop working to make this hope a reality.
	

\end{document}
