\documentclass[a4paper,11pt]{article}
\usepackage[utf8]{inputenc}

\usepackage{longtable}
% extra packages
\usepackage[top=1in, left=1in, right=1in, bottom=1in]{geometry}
\usepackage{amsmath, amssymb}
\usepackage{graphicx}
\usepackage{color}
\usepackage{hyperref}

%\setlength{\parindent}{0pt}

\begin{document}

\begin{center}
\Large \textbf{Diversity Statement}

\Large Youngmin Park
\end{center}

I am a first-generation native South Korean who grew up all over the world (including Swaziland, South Korea, and the Philippines), but by and large grew up in poverty in the United States. These environments exposed me to diverse cultures (including severe socio-economic disparities), and I learned to adapt to many types of foreign etiquette at an early age. While I have lived in the US since adolescence, this country is incredibly unique and I have continuously adjusted over the past several decades to fit into its culture. This experience, combined with my close friendships with people of diverse backgrounds, has helped me understand the difficulty of adjusting to a new life in the US while attending a competitive program. We must actively help students who struggle in this way for the good of the nation's economy and security \cite{jones2018call}.

While growing up, I made many friends with people with vastly different socio-economic and cultural backgrounds. Looking back, it is especially striking that none of my friends from poor families earned a college degree, even though they are all incredibly intelligent. Those of us who went to the same well-funded high school took very different paths, and it is not simply an issue of interest. They are brilliant people who would have thrived in college and could have greatly advanced their careers on the newfound connections. People who could have advanced to positions with greater sway over greater numbers of people, allowing them to guide policies and make decisions benefiting their community, and eventually, their city, state, and country. The world pushed them away from their full potential, and this realization is painful. Indeed, my anecdotal experience aligns with known demographic differences \cite{jones2018call} due to a lack of student engagement \cite{kokkelenberg2010succeeds,savaria2017critical}.

I have taken steps to fight against this unfairness for future generations. From the moment I started teaching at U Pitt, my attempts at fostering diversity, equity, and inclusion have been implicit but powerful. My teaching style takes elements from expeditionary learning, inquiry-based learning, and differentiated instruction. The active component of my teaching is important because it has been shown to improve student performance in underrepresented groups in STEM \cite{theobald2020active}, while differentiated instruction allows me to tailor my teaching to the individual.

My teaching style has always been driven by this principle of viewing students as individuals and not just as a classroom. My teaching efforts are always \textbf{independent of a student's race, gender, or socio-economic background}. I hope to see a world where every student has access to teachers who genuinely want their students learn and succeed, based on the tenets of \textbf{diversity, equity, and inclusion}. To this end I have taught as a guest lecturer in science for underrepresented Bangladeshi children at Moder Patshala and the Free Library of Pennsylvania. At Brandeis, I volunteered in science outreach to give neuroscience lectures at Waltham High School (the series was canceled due to COVID-19).

Clarkson University hosts an unusually comprehensive set of tools to help faculty promote diversity. I wish to participate programs offered by the \textbf{Office for Teaching and Learning} to perfect my ability to teach to a diverse student body. While I independently strive to be exceptionally careful with regards to diversity, it is only through discussion and training that I can identify and overcome my implicit biases. I am especially touched by the efforts by Clarkson University to provide student support in diversity and academics. I have witnessed first-hand the unbearable stress and anxiety that comes with working in a competitive university environment, and would be honored to be a part of an environment that values students' well-being to this degree.

%Racism causes results in preconceived notions formed by exaggerated circumstances such as those that occur in popular media. Every day I work to understand that each individual I meet is no more than my interactions with them, and that I should form my own opinions based on our mutual experiences. This philosophy naturally transfers to my teaching, where I evaluate a student's needs based directly on my interactions with them, as opposed to what I have experienced with people of a similar socio-economic status. It is a lifelong challenge that I wholeheartedly accept. Some students require more mentoring, while others recognize patterns very quickly. \textit{By adjusting my teaching based on the individual, I become a maximally effective teacher}. I help all of my students to the best of my abilities regardless of race, gender, background, and disability.

\newpage
\bibliographystyle{plain}
\bibliography{edi}

\end{document}
