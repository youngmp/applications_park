\documentclass[a4paper,11pt]{article}
\usepackage[utf8]{inputenc}

\usepackage[usenames,dvipsnames]{xcolor}
\usepackage{longtable}
% extra packages
\usepackage[top=1in, left=1in, right=1in, bottom=1in]{geometry}
\usepackage{amsmath, amssymb}
\usepackage{graphicx}
\usepackage{color}
\usepackage{hyperref}
\usepackage{colortbl}

%\setlength{\parindent}{0pt}

\begin{document}

\begin{center}
\Large \textbf{Teaching Statement}

\Large Youngmin Park
\end{center}

\section{Teaching Experience}

\noindent\rule{15cm}{0.4pt}

\textbf{Brandeis (2019-2021)}

\begin{tabular}{p{0.11\linewidth}p{0.4\linewidth}p{.3\linewidth}}
	\textit{Type} & \textit{Class} & \textit{Term(s)}\\
	{Lecture} & Linear Algebra & Spring 2020
\end{tabular}

\noindent\rule{15cm}{0.4pt}

\textbf{University of Pittsburgh (2013-2018)}

\begin{tabular}{p{0.11\linewidth}p{0.4\linewidth}p{.3\linewidth}}
	 Lecture & Differential Equations (3 sections) & Summers, 2014--2017\\
	 & Linear Algebra & Summer 2015 \\
	 & Discrete Math & Spring 2015 \\
	 Recitation & Computational Neuroscience & Summers, 2014--2017 \\
	  & Business Calculus (6 sections) & Fall/Spring 2013/16\\
	  & Calculus 1, 2, 3 (6 sections) & Fall/Spring 2014--2016\\
	 Grading & Differential Equations (10 sections) & Fall/Spring 2013--2017  \\
	 & Complex Variables and Applications & Spring 2017 \\
	 & Linear Algebra (2 sections) & Spring 2016
\end{tabular}

\noindent\rule{15cm}{0.4pt}

\textbf{Oberlin (2013)}

\begin{tabular}{p{0.11\linewidth}p{0.4\linewidth}p{.2\linewidth}}
	 Assistant & Computational Neuroscience  & Winter 2013 
\end{tabular}

\vspace{1cm}

My teaching experience spans eight years at three institutions. My teaching style has consistently led to strong teaching evaluations including a shortlist for the Elizabeth Baranger teaching award, an award which serves to recognize and reward outstanding teaching by graduate students at the University of Pittsburgh.


As the lecturer, I independently designed each course and prepared all materials including lectures, quizzes, tests, and homework assignments. In addition to grading, my teaching duties included meeting students during office hours, making additional appointments as needed. Each semester the class varied in size, ranging from 9 students to as many as 33. 

Another substantial part of my teaching portfolio includes serving as a teaching assistant and leading recitations, where the main lectures were given by a professor. These recitations were for single- and multi-variable calculus classes. In a typical semester, I led three recitation sections per week, where I spent one hour per section teaching calculus concepts (in coordination with the lecturer), and spent another hour working with students in a computer lab. In the lab, students solved automatically-generated calculus problems (generated using Lon Capa), and I provided appropriate hints as they got stuck.

I also led recitations for a course in computational neuroscience. In these recitations, I answered students' questions, and wrote MATLAB scripts on-the-fly to demonstrate simple concepts behind neural models, such as the numerical integration of ordinary differential equations. I also served as the grader for this course, which consisted of 21 students. All recitations were supplemented by office hours and additional appointments as needed.

% The final portion of my teaching portfolio includes grading. In addition to grading as a lecturer and recitation leader, I was often assigned to be a dedicated grader for particular classes, such as introductory differential equations (a one-semester course for engineers), advanced undergraduate differential equations (a two-semester course for math majors), and complex variables. In this role, I acted as a recitation leader, but met with students outside of class to help them better understand the material. These meetings took place during office hours and appointments as needed.



\section{Teaching Philosophy}

\textit{My teaching is fundamentally based on the belief that learning and understanding come with practice and context}. To this end I provide challenges of varying difficulty in the form of assignments and in-class exercises to maximize exposure to the material.

% My lectures complement the textbook by covering missing steps and important applications that may be glossed over in textbooks in favor of technical details.

The first step of familiarizing students with the material is to present and assign sufficient rote problems. These problems are ``plug-and-chug'' applications of formulas, which I believe to be absolutely crucial. In any other profession, improvement is achieved through practice. Musicians play scales and athletes drill. No matter the field, mastery of the most basic skills provides the foundation for advanced study. Through rote study, my students learn the notation, build a foundational understanding, and familiarize themselves with the language on which I build their knowledge.

The second step is to challenge the students. While rote practice is crucial, it is far from a complete learning paradigm. It is important to push students to see the bigger picture and apply fundamental skills in more challenging contexts. Derivatives are straightforward and are important to know, but they are most useful for solving problems such as related rates. Integral rules are good to know, but they are extremely useful for calculating areas and volumes. To aid in my students' understanding, I provide many examples of these applications.

To aid in these two steps, I assign students to work in pairs on straightforward problems and encourage discussion. This active discussion leads to a mutually beneficial give-and-take:  as the students encounter difficulties, they ask each other questions. They often overcome these difficulties autonomously, teaching each other in their own words. \textit{These are the steps I take to provide students with a wealth of practice}.

Learning mathematics without knowing where it came from or when it is used can be dangerously disengaging. To hedge against disengagement, I briefly cover history and applications where appropriate. In my linear algebra classes, I explain that the determinant -- which today is learned as a property of matrices -- was known long before matrices existed. It was the ancient Chinese that discovered determinants, and their mathematicians used the determinant to great effect solving systems of linear equations. Matrices as we know them today were formalized many centuries later, and only with great effort spanning many decades. Indeed, we see the payoff of this effort in the fundamental and ubiquitous usage of matrices and arrays in mathematics, science, engineering, and computer science.

In other lectures, I mention the many uses of eigenvalues in biology, physics, and chemistry. In particular, imaginary eigenvalues with a real component that changes from negative to positive (or vice-versa) plays a role in an incredible number of spatio-temporal dynamics, such as the formation of oscillating cortical waves observed during epileptic seizures, and the formation of patterns on animal hide. \textit{These are the steps I take to provide students with a wealth of context}.

Finally, it is critical to note that adjusting my teaching style through experience is a high priority. As good pedagogical practices become known, I implement them where appropriate. Through this experience, I have found that a combination of rote practice, challenging applications, and context are effective teaching tools. However, I hold myself to high teaching standards, and believe that this process will be a long-long journey.


\section{Summary of Selected Teaching Evaluations}

%\begin{tabular}{p{0.11\linewidth}p{0.4\linewidth}p{.3\linewidth}}
\begin{tabular}{lllll}
	Term & School & Course & Type & \textbf{Teaching Effectiveness (out of 5)}\\
	\hline
	Fall 2014 &U Pitt & Calculus 1 & Recitation &  \textbf{4.5}\\
	Summer 2015 &U Pitt & Linear Algebra & Lecture &  \textbf{4.24}\\
	Summer 2016 &U Pitt & Differential Equations & Lecture &  \textbf{4.4}\\
	Summer 2017 &U Pitt & Differential Equations & Lecture &  \textbf{4.86}\\
	Spring 2020 &Brandeis & Linear Algebra & Lecture &  \textbf{4.7}\\
\end{tabular}
 
\end{document}
