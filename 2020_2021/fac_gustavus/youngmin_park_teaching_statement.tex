\documentclass[a4paper,11pt]{article}
\usepackage[utf8]{inputenc}

\usepackage[usenames,dvipsnames]{xcolor}
\usepackage{longtable}
% extra packages
\usepackage[top=1in, left=1in, right=1in, bottom=1in]{geometry}
\usepackage{amsmath, amssymb}
\usepackage{graphicx}
\usepackage{color}
\usepackage{hyperref}
\usepackage{colortbl}

%\setlength{\parindent}{0pt}

\begin{document}

\begin{center}
\Large \textbf{Teaching Philosophy}

\Large Youngmin Park
\end{center}

\textit{I teach based on the principle that practice and context drive robust learning}. I aim for two goals to this end.

\textbf{In the first goal, I aim to familiarize students with the material by assigning sufficient rote problems}. These problems are straightforward applications of formulas, definitions, and theorems, which I believe to be absolutely crucial to establish a baseline understanding. In any other profession, improvement is achieved through practice. Musicians, athletes, and artists perform drills. No matter the field, mastery of the most basic skills provides the foundation for advanced study. To this end, I assign students to work in pairs on straightforward problems and encourage discussion. This active discussion leads to a mutually beneficial give-and-take:  as the students encounter difficulties, they ask each other questions. They often overcome these difficulties autonomously, teaching each other in their own words.

While rote practice is crucial, it is far from a complete learning paradigm. It is important to push students to see the bigger picture and apply fundamental skills in more challenging contexts. Therefore, \textbf{in the second goal, I aim to challenge the students through challenging applications}. Derivatives are straightforward to execute, but they are useful for solving problems such as related rates. Integral rules are straightforward to memorize, but they are useful for calculating areas and volumes. I show them how derivatives and integrals are fundamentally linked through the Fundamental Theorem of Calculus, which naturally leads into numerical integration methods like forward Euler and Runge-Kutta. General numerical methods naturally lead to discussions on complex biological models that can not be solved by hand, and the students can begin to be exposed to research-level concepts in mathematical modeling. These models may be so complicated that additional methods of analysis must be discussed, such as phase-plane analysis, which naturally leads into the discussion of eigenvectors and eigenvalues.

In brief, my goal as a teacher is not to only present facts and tell my students to memorize them. My goal is to show my students that mathematical tools have a rich history of triumph and defeat. Matrices were not created in a vacuum, but were constructed over the course of decades to provide better notation for the more general concept behind systems of equations. Brilliant minds developed the $\varepsilon$-$\delta$ definition of continuity at a time when humanity had only ill-defined concepts of continuity. The famous Fourier series was originally developed as a method to solve the heat equation in a metal plate. The stories are endless, and discussing them only helps strengthen students' conceptual understanding.

%Learning mathematics without knowing where it came from or when it is used can be dangerously disengaging. To hedge against disengagement, I briefly cover history and applications where appropriate. In my linear algebra classes, I explain that the determinant -- which today is learned as a property of matrices -- was known long before matrices existed. It was the ancient Chinese that discovered determinants, and their mathematicians used the determinant to great effect solving systems of linear equations. Matrices as we know them today were formalized many centuries later, and only with great effort spanning many decades. Indeed, we see the payoff of this effort in the fundamental and ubiquitous usage of matrices and arrays in mathematics, science, engineering, and computer science.

%In other lectures, I mention the many uses of eigenvalues in biology, physics, and chemistry. In particular, imaginary eigenvalues with a real component that changes from negative to positive (or vice-versa) plays a role in an incredible number of spatio-temporal dynamics, such as the formation of oscillating cortical waves observed during epileptic seizures, and the formation of patterns on animal hide. \textit{These are the steps I take to provide students with a wealth of context}.

Adjusting my teaching style through experience is a high priority. As good pedagogical practices become known, I implement them where appropriate. I have found that a combination of rote practice, challenging applications, and context are effective teaching tools. I hold myself to high teaching standards, and believe that the process of becoming a good teacher will be a life-long journey.

% 
 
\end{document}
