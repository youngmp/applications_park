\documentclass[a4paper,11pt]{article}
\usepackage[utf8]{inputenc}

\usepackage{longtable}
% extra packages
\usepackage[top=1in, left=1in, right=1in, bottom=1in]{geometry}
\usepackage{amsmath, amssymb}
\usepackage{graphicx}
\usepackage{color}
\usepackage{hyperref}

%\setlength{\parindent}{0pt}

\begin{document}

\begin{center}
\Large \textbf{Diversity Statement}

\Large Youngmin Park
\end{center}

I am a native South Korean who has lived as a minority for most of his life. I grew up with wealth in poverty-stricken countries such as Swaziland and the Philippines, and spent the latter part of my youth in wealthy nations such as Korea and the United States, while still in poverty. My experiences with poverty and the struggles of low-income peoples has profoundly influenced my conduct in social contexts. In all social situations, ranging from casual to professional, I am intensely aware of how my interactions affect (and do not affect) other people. \textit{I strongly support Gustavus Adolphus College's commitment to diversity}.

In the professional setting, my conscious efforts at equity, diversity, and inclusion primarily occurred during teaching. In my four years of teaching, I had the pleasure of teaching hundreds of students, where classrooms often consisted of a large variety of socio-economic and educational backgrounds. My goal, first and foremost, was to maintain that \textit{I will give equal attention and care to all students}. To this end, I worked hard to keeping my speech patterns and mannerisms consistent between all students. This conscious effort is of great personal importance, because my experiences with racism revealed the demeaning and demoralizing effect of differing or preferential treatment.

As part of my goal for equity, diversity, and inclusion, I always assisted students with additional needs. One student required larger font (13 points at least) due to problems with his vision, so for each exam, I created an additional version with an augmented font size. He, as well as many other students, brought notes from a staff psychiatrist confirming their need for additional testing time, and that their tests be administered outside the classroom at a special testing facility on campus. For each student and for each exam, I emailed this external testing facility the exam with additional details such as the additional time required, and allowed materials such as calculators and notes. In all cases, I allowed myself to be available by phone so that they could ask questions during the exam.

Racism causes results in preconceived notions formed by exaggerated circumstances such as those that occur in popular media. Every day I work to understand that each individual I meet is no more than my interactions with them, and that I should form my own opinions based on our mutual experiences. This philosophy naturally transfers to my teaching, where I evaluate a student's needs based directly on my interactions with them, as opposed to what I have experienced with people of a similar socio-economic status. It is a lifelong challenge that I wholeheartedly accept. Some students require more mentoring, while others recognize patterns very quickly. \textit{By adjusting my teaching based on the individual, I become a maximally effective teacher}.

Finally, good teaching requires a deep commitment that extends beyond the classroom. Thus, I never constrain my teaching to office hours, and often answer questions through email and make additional appointments as needed. I make it a point that this feature is available to all of my students. For all students who seek additional help, I help them to the best of my abilities regardless of race, gender, background, and disability.


\end{document}
