\documentclass[a4paper,11pt]{article}
\usepackage[utf8]{inputenc}

\usepackage{longtable}
% extra packages
\usepackage[top=1in, left=1in, right=1in, bottom=1in]{geometry}
\usepackage{amsmath, amssymb}
\usepackage{graphicx}
\usepackage{color}
\usepackage{hyperref}

%\setlength{\parindent}{0pt}

\begin{document}
	
	\begin{center}
		\Large \textbf{Statement on Commitment to Evidence-Based Free Inquiry and Tolerance to Diverse Ideas}
		
		\Large Youngmin Park
	\end{center}
	
	\begin{itemize}
		\item \textbf{Open-mindedness for diverse ideas including those different from one's own}
		I strongly affirm Oregon State's commitment to open-mindedness for diverse ideas and the encouragement to consider ideas different from one's own. There is no serious progress without the relatively mild risk of \textit{listening}. It is a difficult thing to learn how to listen to anyone of any background, but ideas are not created in a vacuum and do not only come from the status quo. I have always been a strong advocate for listening, and listening primarily for the sake of \textit{understanding}. The latter never follows without the former. Understanding has helped build and maintain relationships in my person and professional life.
		
		\item \textbf{Free inquiry based on evidence and criticism}
		I strongly affirm Oregon State's commitment to free inquiry. The ability to question is often taken for granted but truly sacred. To concretely state the importance of inquiry is an important step towards equality. I am originally from a culture that valued accepting authority without question while demonizing any form of inquiry. It is only through my academic training I've learned to question freely and understand its value. To question based on evidence is a fundamental part of critical thinking and must be promoted as much as possible, in part to counteract the misinformation promoted across the internet.
	\end{itemize}

	%\newpage
	%\bibliographystyle{plain}
	%\bibliography{edi}
	
\end{document}
