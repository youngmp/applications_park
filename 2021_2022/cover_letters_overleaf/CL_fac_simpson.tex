%%%%%%%%%%%%%%%%%%%%%%%%%%%%%%%%%%%%%%%%%%%%%%%%%%%%%%%%
%%%%%%%%%%%------RMIT Letterhead------------%%%%%%%%%%%%
%%%%%%%%%%%%%%%%%%%%%%%%%%%%%%%%%%%%%%%%%%%%%%%%%%%%%%%%
\documentclass[11pt,a4paper]{letter}

%\usepackage[english]{babel}
%\usepackage[utf8x]{inputenc}
%\usepackage[utf8]{inputenc}

%\usepackage[T1]{fontenc}

% you may specify the font size (10pt, 11pt and 12pt) and paper size (letterpaper, a4paper, etc.)
\usepackage{microtype}
\usepackage{graphicx}
\usepackage{gfsdidot}
\usepackage[T1]{fontenc}

\documentclass[11pt,a4paper]{letter}

\usepackage{microtype}
\usepackage{graphicx}
\usepackage{gfsdidot}
\usepackage[T1]{fontenc}
\usepackage{geometry}
\geometry{
	a4paper,
	left=20mm,
	right=20mm,
	headheight=.0cm,
	top=.0cm,
	bottom=0cm,
	footskip=0cm
}




%----------------------------------------------------------------------------------------
%	SENDER INFORMATION
%----------------------------------------------------------------------------------------

\def\Who{Youngmin Park, Ph.D.}
\def\What{, PIMS Postdoctoral Fellow}
\def\Where{Department of Mathematics\\University of Manitoba}
\def\addy{420 Machray Hall 186 Dysart Rd}
\def\CityZip{Winnipeg, MB R3T 2N2 Canada\\\today}
\def\Email{E-mail: ypark@brandeis.edu}
\def\TEL{Phone: (412) 805-0283}
\def\URL{youngmp.github.io}
%----------------------------------------------------------------------------------------
%	HEADER AND FROM ADDRESS STRUCTURE
%----------------------------------------------------------------------------------------
%%%%
%\vspace{-2in}
\def\address{
	\raisebox{0in}[0pt][0pt]{\includegraphics[width=1.2in]{logo_manitoba.png}} % Include the logo of your institution
	\hspace{-3.31in} % Position of the institution logo, increase to move left, decrease to move right
	\vskip -.35in~\\ % Position of the text in relation to the institution logo, increase to move down, decrease to move up
	%\makebox[0ex][r]{\bf \Who \What }\hspace{0.08in}
	~\\[-0.11in] % Reduce the whitespace above the horizontal rule
	\hspace{\fill}\parbox[t]{2.85in}{ % Create a box for your details underneath the horizontal rule on the right
		\footnotesize % Use a smaller font size for the details
		\Who \\ \em % Your name, all text after this will be italicized
		\Where\\ % Your department
		\addy\\ % Your address
		\CityZip\\ % Your city and zip code
		%\TEL\\ % Your phone number
		%\Email\\ % Your email address
		%\URL % Your URL
	}
	\hspace{-1in} % Horizontal position of this block, increase to move left, decrease to move right
	\vspace{-1.6in} % Move the letter content up for a more compact look
}


%----------------------------------------------------------------------------------------
%	TO ADDRESS STRUCTURE
%----------------------------------------------------------------------------------------


\def\opening#1{\thispagestyle{empty}
	%%% from address
	{\centering\address \vspace{.75in} \\
		%% Date
		\hspace*{\longindentation}\hspace*{\fill}\par} % remove \today to not display it
	{\raggedright \toname \\ \toaddress \par} % Print the to name and address
	
	\noindent #1 % Print the opening line
}


%----------------------------------------------------------------------------------------
%	SIGNATURE
%----------------------------------------------------------------------------------------

\signature{\Who \What}

\long\def\closing#1{
	\vspace{0.1in} % Some whitespace after the letter content and before the signature
	\noindent % Stop paragraph indentation
	#1 % Print the signature text
	
	\includegraphics[width=1in]{signature.png} % Whitespace between the signature text and your name%
	
	\fromsig}


%\long\def\closing#1{
%\vspace{0.1in} % Some whitespace after the letter content and before the signature
%\noindent % Stop paragraph indentation
%\hspace*{\longindentation} % Move the signature right
%\parbox{\indentedwidth}{\raggedright
%#1 % Print the signature text
%
%\includegraphics[width=1in]{signature.png} % Whitespace between the signature text and your name%
%
%\fromsig}}

%----------------------------------------------------------------------------------------
% Create a new command for the horizontal rule in the document which allows thickness specification
\makeatletter
\def\vhrulefill#1{\leavevmode\leaders\hrule\@height#1\hfill \kern\z@}
\makeatother



%----------------------------------------------------------------------------------------
%	PAPER MARGINS
%----------------------------------------------------------------------------------------

\textwidth 6in
\textheight 11in
%\oddsidemargin -.4in
\oddsidemargin 0in
\evensidemargin 0in
\topmargin -1.2in
\longindentation 0.25\textwidth
\parindent 0.2in


\begin{document}

%----------------------------------------------------------------------------------------
%	TO ADDRESS
%----------------------------------------------------------------------------------------

\def\School{Simpson College}

\begin{letter}
{Search Committee\\
Mathematics Department\\
701 N C Street\\
Indianola IA 50125}

%----------------------------------------------------------------------------------------
%	LETTER CONTENT
%----------------------------------------------------------------------------------------

\opening{Dear Members of the Search Committee,}


I am applying for the position of Assistant Professor at the \School~Department of Mathematics. I hold a Ph.D. in mathematics from the University of Pittsburgh advised by G. Bard Ermentrout, and now hold a postdoctoral position at Brandeis University advised by Thomas G. Fai. I specialize in \textbf{deterministic and stochastic dynamical systems}, \textbf{bifurcation theory}, \textbf{perturbation methods}, and \textbf{non-local integro-differential equations}, with applications to neuroscience and cell physiology.

My doctoral research in dimension reduction resulted in winning the prestigious \textbf{Andrew Mellon Predoctoral Fellowship}, awarded to doctoral students of exceptional promise and ability. My research ability has matured through my postdocs. At the University of Pennsylvania, I collaborated closely with neuroscientists and published ground-breaking neural models. At Brandeis, I have continued to develop my abilities as an independent mathematician and published promising new methods in fields such as coupled oscillators and molecular motor dynamics.

My research and teaching make for valuable, healthy contributions to an undergraduate institution. Virtually any undergraduate who learns calculus and a programming language will have the opportunity to take the lead in publishable research projects in math and biology. They will learn the principles of scientific thought, which will serve them well at \School~and beyond.

I am committed to providing high-quality, equal, and accessible education for all students, as evidenced by my teaching evaluations. My teaching portfolio boasts eight years of teaching at different levels (calculus sequence, differential equations, linear algebra, and discrete math). As a doctoral student, I was shortlisted for the \textbf{Elizabeth Baranger teaching award}, the most prestigious teaching award at the University of Pittsburgh. I strongly believe in the principle that learning is best accomplished with context and practice. My students do not learn in a vacuum, but instead learn concepts alongside the human triumph and defeat behind many concepts we take for granted today, such as matrix notation and Euclid's fifth postulate. Students at \School~will greatly benefit from this approach, and in turn, will drive me towards excellent instruction. I believe that students are best taught based on the needs of the individual and use numerous teaching styles to this end (including expeditionary learning, inquiry-based learning, and differentiated instruction).

Finally, I am a first-generation US Citizen who grew up all over the world  (including Swaziland, South Korea, and the Philippines), but by and large grew up in poverty in the United States. These environments exposed me to diverse cultures (including severe socio-economic disparities). While I have lived in the US since adolescence, this country is incredibly unique and I have continuously adjusted over the past several decades to fit into its culture. This experience, combined with my close friendships with people of diverse backgrounds, has helped me understand the difficulty of adjusting to a new life in the US while attending a competitive program. I believe that I could make valuable leadership contributions to diversity programs at \School.


As part of my application I include a curriculum vitae, teaching statement, and research statement. Please request additional details as needed, and I look forward to our correspondence.


\closing{Sincerely,}
\end{letter}
\end{document}